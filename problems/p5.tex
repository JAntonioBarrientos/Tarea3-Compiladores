\begin{Pro}
De las siguientes expresiones, realiza sus cuádruples, triples y triples indirectos:
\begin{enumerate}
    \item[(a)] $a+(b+c)$
    \item[(b)] $-(a+b)*(c+d)$
    \item[(c)] $a[i] = b*c - d*c$
    \item[(d)] $x = a[i] + b[j]$
    \item[(e)] $x = f(x+1, y) + 2$
\end{enumerate}    
\end{Pro}


\begin{enumerate}
    \item[(a)] $a+(b+c)$

El orden de evaluación es que primero se debe calcular la subexpresión dentro del paréntesis $(b+c)$ y luego sumar ese resultado con $a$.

\subsection*{1. Cuádruples}
En los cuádruples utilizamos variables temporales explícitas ($t_1, t_2$) para almacenar los resultados intermedios.

\begin{center}
\begin{tabular}{|c|c|c|c|c|}
    \hline
    \textbf{Instrucción} & \textbf{Op} & \textbf{Arg1} & \textbf{Arg2} & \textbf{Resultado} \\
    \hline
    (1) & $+$ & $b$ & $c$ & $t_1$ \\
    \hline
    (2) & $+$ & $a$ & $t_1$ & $t_2$ \\
    \hline
\end{tabular}
\end{center}
\textit{Interpretación: Primero $t_1 = b + c$, luego $t_2 = a + t_1$.}

\subsection*{2. Triples}
En los triples no definimos $t_n$. Referenciamos el resultado de la suma anterior usando su índice (posición) entre paréntesis, en este caso $(0)$ refiere al resultado de la primera operación.

\begin{center}
\begin{tabular}{|c|c|c|c|}
    \hline
    \textbf{Índice} & \textbf{Op} & \textbf{Arg1} & \textbf{Arg2} \\
    \hline
    (0) & $+$ & $b$ & $c$ \\
    \hline
    (1) & $+$ & $a$ & $(0)$ \\
    \hline
\end{tabular}
\end{center}

\subsection*{3. Triples Indirectos}
Se utilizan dos tablas. La tabla de instrucciones contiene apuntadores a la tabla de triples.

\textbf{Tabla de Instrucciones (Apuntadores)}
\begin{center}
\begin{tabular}{|c|c|}
    \hline
    \textbf{Posición} & \textbf{Apunta a Triple} \\
    \hline
    35 & (0) \\
    \hline
    36 & (1) \\
    \hline
\end{tabular}
\end{center}

\textbf{Tabla de Triples}
\begin{center}
\begin{tabular}{|c|c|c|c|}
    \hline
    \textbf{ID} & \textbf{Op} & \textbf{Arg1} & \textbf{Arg2} \\
    \hline
    0 & $+$ & $b$ & $c$ \\
    \hline
    1 & $+$ & $a$ & $(0)$ \\
    \hline
\end{tabular}
\end{center}
\end{enumerate}    

