\begin{Pro}
De las siguientes expresiones, realiza sus cuádruples, triples y triples indirectos:
\begin{enumerate}
    \item[(a)] $a+(b+c)$
    \item[(b)] $-(a+b)*(c+d)$
    \item[(c)] $a[i] = b*c - d*c$
    \item[(d)] $x = a[i] + b[j]$
    \item[(e)] $x = f(x+1, y) + 2$
\end{enumerate}    
\end{Pro}


\begin{enumerate}
    \item[(a)] $a+(b+c)$

El orden de evaluación es que primero se debe calcular la subexpresión dentro del paréntesis $(b+c)$ y luego sumar ese resultado con $a$.

\subsection*{1. Cuádruples}
En los cuádruples utilizamos variables temporales explícitas ($t_1, t_2$) para almacenar los resultados intermedios.

\begin{center}
\begin{tabular}{|c|c|c|c|}
    \hline
    \textbf{Op} & \textbf{Arg1} & \textbf{Arg2} & \textbf{Resultado} \\
    \hline
     $+$ & $b$ & $c$ & $t_1$ \\
    \hline
     $+$ & $a$ & $t_1$ & $t_2$ \\
    \hline
\end{tabular}
\end{center}
\textit{Interpretación: Primero $t_1 = b + c$, luego $t_2 = a + t_1$.}

\subsection*{2. Triples}
En los triples no definimos $t_n$. Referenciamos el resultado de la suma anterior usando su índice (posición) entre paréntesis, en este caso $(0)$ refiere al resultado de la primera operación.

\begin{center}
\begin{tabular}{|c|c|c|c|}
    \hline
    \textbf{Índice} & \textbf{Op} & \textbf{Arg1} & \textbf{Arg2} \\
    \hline
    (0) & $+$ & $b$ & $c$ \\
    \hline
    (1) & $+$ & $a$ & $(0)$ \\
    \hline
\end{tabular}
\end{center}

\subsection*{3. Triples Indirectos}
Se utilizan dos tablas. La tabla de instrucciones contiene apuntadores a la tabla de triples.

\textbf{Tabla de Instrucciones (Apuntadores)}
\begin{center}
\begin{tabular}{|c|c|}
    \hline
    \textbf{Posición} & \textbf{Apunta a Triple} \\
    \hline
    35 & (0) \\
    \hline
    36 & (1) \\
    \hline
\end{tabular}
\end{center}

\textbf{Tabla de Triples}
\begin{center}
\begin{tabular}{|c|c|c|c|}
    \hline
    \textbf{ID} & \textbf{Op} & \textbf{Arg1} & \textbf{Arg2} \\
    \hline
    0 & $+$ & $b$ & $c$ \\
    \hline
    1 & $+$ & $a$ & $(0)$ \\
    \hline
\end{tabular}
\end{center}




\item[(b)] $-(a+b)*(c+d)$

El orden de evaluación sigue la jerarquía de operadores: primero paréntesis, luego operadores unarios, y al final la multiplicación.

\subsection*{1. Cuádruples}

\begin{center}
\begin{tabular}{|c|c|c|c|}
    \hline
    \textbf{Op} & \textbf{Arg1} & \textbf{Arg2} & \textbf{Resultado} \\
    \hline
    $+$ & $a$ & $b$ & $t_1$ \\
    \hline
    \texttt{minus} & $t_1$ & - & $t_2$ \\
    \hline
    $+$ & $c$ & $d$ & $t_3$ \\
    \hline
    $*$ & $t_2$ & $t_3$ & $t_4$ \\
    \hline
\end{tabular}
\end{center}

\subsection*{2. Triples}


\begin{center}
\begin{tabular}{|c|c|c|c|}
    \hline
    \textbf{Índice} & \textbf{Op} & \textbf{Arg1} & \textbf{Arg2} \\
    \hline
    (0) & $+$ & $a$ & $b$ \\
    \hline
    (1) & \texttt{minus} & (0) & - \\
    \hline
    (2) & $+$ & $c$ & $d$ \\
    \hline
    (3) & $*$ & (1) & (2) \\
    \hline
\end{tabular}
\end{center}

\subsection*{3. Triples Indirectos}


\textbf{Tabla de Instrucciones}
\begin{center}
\begin{tabular}{|c|}
    \hline
    \textbf{Apunta a Triple} \\
    \hline
    (0) \\
    \hline
    (1) \\
    \hline
    (2) \\
    \hline
    (3) \\
    \hline
\end{tabular}
\end{center}

\textbf{Tabla de Triples}
\begin{center}
\begin{tabular}{|c|c|c|c|}
    \hline
    \textbf{ID} & \textbf{Op} & \textbf{Arg1} & \textbf{Arg2} \\
    \hline
    0 & $+$ & $a$ & $b$ \\
    \hline
    1 & \texttt{minus} & (0) & - \\
    \hline
    2 & $+$ & $c$ & $d$ \\
    \hline
    3 & $*$ & (1) & (2) \\
    \hline
\end{tabular}
\end{center}


\item[(c)] $a[i] = b*c - d*c$

La operación $a[i]$ indica un acceso a la posición $i$ del arreglo $a$. Dado que está a la izquierda de la asignación, es una operación de escritura (almacenamiento).

\subsection*{1. Cuádruples}
Generamos temporales para las multiplicaciones y la resta. Para la asignación al arreglo, utilizamos el operador especial \texttt{[]=} donde \texttt{Arg1} es el arreglo, \texttt{Arg2} es el índice y \texttt{Resultado} es el valor a guardar.
\begin{center}
\begin{tabular}{|c|c|c|c|}
    \hline
    \textbf{Op} & \textbf{Arg1} & \textbf{Arg2} & \textbf{Resultado} \\
    \hline
    $*$ & $b$ & $c$ & $t_1$ \\
    \hline
    $*$ & $d$ & $c$ & $t_2$ \\
    \hline
    $-$ & $t_1$ & $t_2$ & $t_3$ \\
    \hline
    \texttt{[]=} & $a$ & $i$ & $t_3$ \\
    \hline
\end{tabular}
\end{center}

\subsection*{2. Triples}
Como los triples solo tienen dos argumentos (\texttt{Arg1}, \texttt{Arg2}), no podemos expresar la asignación de arreglo en una sola línea ($a$, $i$, $val$). Lo descomponemos en dos pasos lógicos:
1. Calcular la referencia a $a[i]$ (operador \texttt{[]}).
2. Asignar el valor a esa referencia (operador \texttt{assign} o \texttt{=}).

\begin{center}
\begin{tabular}{|c|c|c|c|}
    \hline
    \textbf{Índice} & \textbf{Op} & \textbf{Arg1} & \textbf{Arg2} \\
    \hline
    (0) & $*$ & $b$ & $c$ \\
    \hline
    (1) & $*$ & $d$ & $c$ \\
    \hline
    (2) & $-$ & (0) & (1) \\
    \hline
    (3) & \texttt{[]} & $a$ & $i$ \\
    \hline
    (4) & \texttt{assign} & (3) & (2) \\
    \hline
\end{tabular}
\end{center}
\textit{Nota: En la instrucción (3) obtenemos la dirección de $a[i]$ y en la (4) guardamos el resultado de la resta (2) en esa dirección.}

\subsection*{3. Triples Indirectos}

\textbf{Tabla de Instrucciones}
\begin{center}
\begin{tabular}{|c|}
    \hline
    \textbf{Apunta a Triple} \\
    \hline
    (0) \\
    \hline
    (1) \\
    \hline
    (2) \\
    \hline
    (3) \\
    \hline
    (4) \\
    \hline
\end{tabular}
\end{center}

\textbf{Tabla de Triples}
\begin{center}
\begin{tabular}{|c|c|c|c|}
    \hline
    \textbf{ID} & \textbf{Op} & \textbf{Arg1} & \textbf{Arg2} \\
    \hline
    0 & $*$ & $b$ & $c$ \\
    \hline
    1 & $*$ & $d$ & $c$ \\
    \hline
    2 & $-$ & (0) & (1) \\
    \hline
    3 & \texttt{[]} & $a$ & $i$ \\
    \hline
    4 & \texttt{assign} & (3) & (2) \\
    \hline
\end{tabular}
\end{center}




\item[(d)] $x = a[i] + b[j]$


\subsection*{1. Cuádruples}


\begin{center}
\begin{tabular}{|c|c|c|c|}
    \hline
    \textbf{Op} & \textbf{Arg1} & \textbf{Arg2} & \textbf{Resultado} \\
    \hline
    \texttt{=[]} & $a$ & $i$ & $t_1$ \\
    \hline
    \texttt{=[]} & $b$ & $j$ & $t_2$ \\
    \hline
    $+$ & $t_1$ & $t_2$ & $t_3$ \\
    \hline
    $=$ & $t_3$ & - & $x$ \\
    \hline
\end{tabular}
\end{center}
\textit{Nota: La última instrucción asigna el valor del temporal $t_3$ a la variable $x$.}

\subsection*{2. Triples}
En la notación de triples, el operador \texttt{=[]} devuelve el valor almacenado en esa posición.

\begin{center}
\begin{tabular}{|c|c|c|c|}
    \hline
    \textbf{Índice} & \textbf{Op} & \textbf{Arg1} & \textbf{Arg2} \\
    \hline
    (0) & \texttt{=[]} & $a$ & $i$ \\
    \hline
    (1) & \texttt{=[]} & $b$ & $j$ \\
    \hline
    (2) & $+$ & (0) & (1) \\
    \hline
    (3) & \texttt{assign} & $x$ & (2) \\
    \hline
\end{tabular}
\end{center}

\subsection*{3. Triples Indirectos}

\textbf{Tabla de Instrucciones}
\begin{center}
\begin{tabular}{|c|}
    \hline
    \textbf{Apunta a Triple} \\
    \hline
    (0) \\
    \hline
    (1) \\
    \hline
    (2) \\
    \hline
    (3) \\
    \hline
\end{tabular}
\end{center}

\textbf{Tabla de Triples}
\begin{center}
\begin{tabular}{|c|c|c|c|}
    \hline
    \textbf{ID} & \textbf{Op} & \textbf{Arg1} & \textbf{Arg2} \\
    \hline
    0 & \texttt{=[]} & $a$ & $i$ \\
    \hline
    1 & \texttt{=[]} & $b$ & $j$ \\
    \hline
    2 & $+$ & (0) & (1) \\
    \hline
    3 & \texttt{assign} & $x$ & (2) \\
    \hline
\end{tabular}
\end{center}




\item[(e)] $x = f(x+1, y) + 2$
\begin{enumerate}
    \item Calcular el argumento $x+1$.
    \item Pasar los parámetros ($t_1$ y $y$) a la función.
    \item Llamar a la función $f$.
    \item Sumar 2 al resultado de la función.
    \item Asignar el total a $x$.
\end{enumerate}

\subsection*{1. Cuádruples}
Utilizamos la instrucción \texttt{param} para preparar los argumentos. La instrucción \texttt{call} suele tener la estructura: \texttt{call nombre\_función, num\_args, variable\_retorno}.

\begin{center}
\begin{tabular}{|c|c|c|c|}
    \hline
    \textbf{Op} & \textbf{Arg1} & \textbf{Arg2} & \textbf{Resultado} \\
    \hline
    $+$ & $x$ & $1$ & $t_1$ \\
    \hline
    \texttt{param} & $t_1$ & - & - \\
    \hline
    \texttt{param} & $y$ & - & - \\
    \hline
    \texttt{call} & $f$ & $2$ & $t_2$ \\
    \hline
    $+$ & $t_2$ & $2$ & $t_3$ \\
    \hline
    $=$ & $t_3$ & - & $x$ \\
    \hline
\end{tabular}
\end{center}

\subsection*{2. Triples}
En los triples, la instrucción \texttt{call} devuelve implícitamente un valor que puede ser referenciado por su índice (en este caso, el índice (3)).

\begin{center}
\begin{tabular}{|c|c|c|c|}
    \hline
    \textbf{Índice} & \textbf{Op} & \textbf{Arg1} & \textbf{Arg2} \\
    \hline
    (0) & $+$ & $x$ & $1$ \\
    \hline
    (1) & \texttt{param} & (0) & - \\
    \hline
    (2) & \texttt{param} & $y$ & - \\
    \hline
    (3) & \texttt{call} & $f$ & $2$ \\
    \hline
    (4) & $+$ & (3) & $2$ \\
    \hline
    (5) & \texttt{assign} & $x$ & (4) \\
    \hline
\end{tabular}
\end{center}

\subsection*{3. Triples Indirectos}

\textbf{Tabla de Instrucciones}
\begin{center}
\begin{tabular}{|c|}
    \hline
    \textbf{Apunta a Triple} \\
    \hline
    (0) \\
    \hline
    (1) \\
    \hline
    (2) \\
    \hline
    (3) \\
    \hline
    (4) \\
    \hline
    (5) \\
    \hline
\end{tabular}
\end{center}

\textbf{Tabla de Triples}
\begin{center}
\begin{tabular}{|c|c|c|c|}
    \hline
    \textbf{ID} & \textbf{Op} & \textbf{Arg1} & \textbf{Arg2} \\
    \hline
    0 & $+$ & $x$ & $1$ \\
    \hline
    1 & \texttt{param} & (0) & - \\
    \hline
    2 & \texttt{param} & $y$ & - \\
    \hline
    3 & \texttt{call} & $f$ & $2$ \\
    \hline
    4 & $+$ & (3) & $2$ \\
    \hline
    5 & \texttt{assign} & $x$ & (4) \\
    \hline
\end{tabular}
\end{center}

\end{enumerate}    
