\begin{Pro}
Construye el DAG para las siguientes expresiones con base en la gramática de expresiones aritméticas:
\begin{enumerate}
    \item[(a)] $a+b*(a+b)$
    \item[(b)] $((x+y)-((x+y)^{*}(x+y)))+((x+y)^{*}(x-y))$
    \item[(c)] $x^{*}(x+x)$
\end{enumerate}
\end{Pro}

Definiremos  la gramática libre de contexto para expresiones aritméticas junto con las reglas semánticas para la construcción del Árbol de Sintaxis Abstracta (AST):


\textbf{Nota:} 
\begin{center}
\renewcommand{\arraystretch}{1.5}
\begin{tabular}{l @{\hspace{1.5cm}} l}
    \toprule
    \textbf{Producción} & \textbf{Reglas Semánticas (Construcción del AST)} \\
    \midrule
    % Nivel Expresión (Sumas y Restas)
    $E \to E_1 + T$ & $E.node = \text{new Node}(\text{'+'}, E_1.node, T.node)$ \\
    $E \to E_1 - T$ & $E.node = \text{new Node}(\text{'-'}, E_1.node, T.node)$ \\
    $E \to T$       & $E.node = T.node$ \\
    \midrule
    % Nivel Término (Multiplicación)
    $T \to T_1 * F$ & $T.node = \text{new Node}(\text{'*'}, T_1.node, F.node)$ \\
    $T \to F$       & $T.node = F.node$ \\
    \midrule
    % Nivel Factor (Paréntesis y básicos)
    $F \to (E)$     & $F.node = E.node$ \\
    $F \to \textbf{id}$ & $F.node = \text{new Leaf}(\textbf{id}, \text{getname}(\textbf{id}))$ \\
    $F \to \textbf{num}$ & $F.node = \text{new Leaf}(\textbf{num}, \text{getvalue}(\textbf{num}))$ \\
    \bottomrule
\end{tabular}
\end{center}
\footnotesize{Los subíndices (ej. $E_1, T_1$) denotan la instancia recursiva del no-terminal para diferenciarla en la regla semántica.}
\normalsize

Para construir el DAG, seguimos un proceso \textit{bottom-up}. Identificamos las hojas (variables/números) únicas y reutilizamos nodos intermedios si la operación y sus operandos son idénticos a uno existente.



\begin{enumerate}
    \item[(a)] $a+b*(a+b)$
    
    \begin{center}
        % Ajusta el scale o width según el tamaño de tu imagen
        \includegraphics[width=0.6\textwidth]{images/p1/p1_a_asa.jpg}
        %caption
        \captionof{figure}{ASA para la expresión $a+b*(a+b)$.}
        \label{fig:p1_a_asa}
    \end{center}

       \textbf{Análisis:}
    \begin{itemize}
        \item Las hojas \textbf{id:a} y \textbf{id:b} son únicas; se crean una sola vez.
        \item La subexpresión $(a+b)$ se calcula primero.
        \item El nodo final suma $a$ con el resultado de la multiplicación.
    \end{itemize}

    \begin{center}
    \begin{tabular}{c c c c l}
        \toprule
        \textbf{ID} & \textbf{Op} & \textbf{Izq} & \textbf{Der} & \textbf{Subexpresión} \\
        \midrule
        1 & id & - & - & $a$ \\
        2 & id & - & - & $b$ \\
        3 & + & 1 & 2 & $a+b$ \\
        4 & * & 2 & 3 & $b * (a+b)$ \\
        5 & + & 1 & 4 & $a + (b*(a+b))$ \\
        \bottomrule
    \end{tabular}
    \end{center}

        \begin{center}
        % Ajusta el scale o width según el tamaño de tu imagen
        \includegraphics[width=0.6\textwidth]{images/p1/p1_a_dag.jpg}
        %caption
        \captionof{figure}{DAG para la expresión $a+b*(a+b)$.}
        \label{fig:p1_a_dag}
    \end{center}


    \item[(b)] $((x+y)-((x+y)^{*}(x+y)))+((x+y)^{*}(x-y))$
    
    \begin{center}
        \includegraphics[width=0.9\textwidth]{images/p1/p1_b_asa.jpg}
        %caption
        \captionof{figure}{ASA para la expresión $((x+y)-((x+y)^{*}(x+y)))+((x+y)^{*}(x-y))$.}
        \label{fig:p1_b_asa}
    \end{center}

 \textbf{Análisis:}
    \begin{itemize}
        \item Identificamos la subexpresión común $S_1 = (x+y)$, la cual se repite 4 veces pero solo genera \textbf{1 nodo} en el DAG.
        \item El nodo $S_1$ es reutilizado como operando para la resta izquierda, la multiplicación central y la multiplicación derecha.
    \end{itemize}

    \begin{center}
    \begin{tabular}{c c c c l}
        \toprule
        \textbf{ID} & \textbf{Op} & \textbf{Izq} & \textbf{Der} & \textbf{Subexpresión / Nota} \\
        \midrule
        1 & id & - & - & $x$ \\
        2 & id & - & - & $y$ \\
        3 & + & 1 & 2 & $(x+y)$ \textit{(Se reutiliza 4 veces)} \\
        4 & * & 3 & 3 & $(x+y)*(x+y)$ \\
        5 & - & 3 & 4 & Parte Izq: $(x+y) - (\text{nodo } 4)$ \\
        6 & - & 1 & 2 & $(x-y)$ \\
        7 & * & 3 & 6 & Parte Der: $(x+y) * (x-y)$ \\
        8 & + & 5 & 7 & \textbf{Raíz} \\
        \bottomrule
    \end{tabular}
    \end{center}

    \begin{center}
        \includegraphics[width=0.9\textwidth]{images/p1/p1_b_dag.jpg}
        %caption
        \captionof{figure}{DAG para la expresión $((x+y)-((x+y)^{*}(x+y)))+((x+y)^{*}(x-y))$.}
        \label{fig:p1_b_dag}
    \end{center}

    \item[(c)] $x^{*}(x+x)$
    

    \begin{center}
        \includegraphics[width=0.6\textwidth]{images/p1/p1_c_asa.jpg}
        %caption
        \captionof{figure}{ASA para la expresión $x^{*}(x+x)$.}
        \label{fig:p1_c_asa}
    \end{center}


     \textbf{Análisis:}
    \begin{itemize}
        \item Solo existe una variable $x$. Se crea un único nodo hoja para ella.
        \item El nodo hoja $x$ tendrá 3 padres en el grafo (dos conexiones hacia el $+$ y una hacia el $*$).
    \end{itemize}

    \begin{center}
    \begin{tabular}{c c c c l}
        \toprule
        \textbf{ID} & \textbf{Op} & \textbf{Izq} & \textbf{Der} & \textbf{Subexpresión} \\
        \midrule
        1 & id & - & - & $x$ \\
        2 & + & 1 & 1 & $x+x$ \\
        3 & * & 1 & 2 & $x * (x+x)$ \\
        \bottomrule
    \end{tabular}
    \end{center}

    \begin{center}
        \includegraphics[width=0.6\textwidth]{images/p1/p1_c_dag.jpg}
        %caption
        \captionof{figure}{DAG para la expresión $x^{*}(x+x)$.}
        \label{fig:p1_c_dag}
    \end{center}
\end{enumerate}


